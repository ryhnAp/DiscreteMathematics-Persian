\begin{PROBLEM}
    \p
    در فروشگاهی برای خرید از پول 1 یا 2 یا 5 یا 10 تومانی می‌توان استفاده کرد که
    استفاده همزمان از 5 و 10 تومانی ممکن نیست.
    تابع مولد تعداد حالات ممکن برای خرید یک جنس در این فروشگاه را بیابید. ترتیب پرداخت مهم نیست.


    \SOLUTION{
        \p
        دو حالت برای پرداخت وجود دارد. اول این که از سکه ۵ تومانی استفاده شود و دوم این که نشود. در حالت اول پرداخت‌ها می‌توانند به حالات گوناگون از تابع مولد زیر حاصل شوند:
$$ (1+x+x^2+...)(1+x^2+x^4+...)(x^5+x^{10}+...) $$
در حالت دوم نیز از تابع مولد زیر استفاده می‌شود که می‌تواند در آن سکه‌ی ۱۰ تومانی وجود داشته و یا وجود نداشته باشد:
$$ (1+x+x^2+...)(1+x^2+x^4+...)(1 + x^{10}+x^{20}+...) $$
جواب نهایی از جمع دو عبارت بالا بدست می‌آید که اگر x را کوچکتر از ۱ در نظر بگیریم، با استفاده از 
سری هندسی
 می‌توان جملات بالا را ساده‌تر نوشت که حاصل می‌شود:
$$ \frac{1}{1-x} \times \frac{1}{1-x^2} \times ( (\frac{1}{1-x^5} - 1) + \frac{1}{1-x^{10}} ) $$
    }
\end{PROBLEM}