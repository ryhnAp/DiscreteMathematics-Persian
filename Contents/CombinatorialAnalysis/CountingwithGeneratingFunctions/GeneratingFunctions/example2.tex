\begin{PROBLEM}
    \p
    اگر به تعداد کافی سکه‌ی ۱ و ۲ و ۵ دلاری داشته باشیم، به چند طریق می‌توان یک کالای ۷ دلاری را از دستگاه فروش خرید به طوری که ترتیب انداختن سکه‌ها در دستگاه مهم باشد؟

    \SOLUTION{
        \p
        اگر 
        $n$
        ، تعداد کل سکه‌هایی که داخل دستگاه انداخته‌ایم باشد، تعداد حالت‌های انداختن سکه تا رسیدن به ۷ دلار با ضریب
        $x^{7}$
           در بسط 
        $$(x + x^{2} + x^{5})^{n}$$
            برابر است. زیرا این بسط دارای 
            $n$
             عبارت 
             $$(x + x^{2} + x^{5})$$
              است که در هم ضرب شده‌اند و ضریب
              $x^{7}$
         برابر است با تعداد حالت‌هایی که حاصل جمع توان جمله‌های 
         $x$
          از این عبارت‌ها برابر ۷ شود.      
               \p
             محدودیتی برای تعداد سکه‌های مصرفی نداریم (بجز اینکه مجموع ارزش آن‌ها از ۷ فراتر نرود که طبق روش بالا لحاظ می‌شود). پس با جمع کردن تعداد جواب‌ها به ازای تمام
            $n$‌های
            ممکن، پاسخ مسئله به دست می‌آید. به عبارت دیگر پاسخ مسئله با ضریب 
            $x^{7}$
             در عبارت زیر برابر می‌شود:
                $$1 + (x + x^{2} + x^{5}) + (x + x^{2} + x^{5})^{2} + ...$$
            \p
        با توجه به اینکه در توان‌های کمتر از ۲ و بیشتر از ۷، جمله‌ی 
        $x^{7}$
         را نداریم، از بررسی و محاسبه‌ی آن‌ها در عبارت چند‌جمله‌ای بالا صرف نظر می‌کنیم. با محاسبه‌ی عبارت، ضریب 
        $x^{7}$
        برابر ۲۶ به دست می‌آید.
    }
\end{PROBLEM}

\NOTE{
    توجه کنید در سوال بالا تاکید شده که ترتیب انداختن سکه‌ها با اهمیت است،
    پس با یک سوال ترتیب با اعضای تکراری (
        \CROSSREF[ترکیب چندگانه]{نوشتار کامل ترکیب چندگانه}
        ) مواجه هستیم. نحوه درنظر گرفته شدن این شرط
    در پاسخ بالا حائز اهمیت است. به عنوان مثال، زمانی را درنظر بگیرید که $n=2$ باشد.
    در این صورت، با توجه به اهمیت داشتن ترتیب، دو پاسخ $\{2,5\}$ و $\{5,2\}$ متمایزاند.
    پاسخ این مسئله نیز این دو حالت را متمایز مورد توجه قرار می‌دهد چرا که:
    $$(x + x^2 + x^5)^2 = (x + x^2 + x^5) \times (x + x^2 + x^5)$$
    $$= \dots + x^2 \times x^5 + x^5 \times x^2 + ... = \dots + 2x^7 + \dots$$
}

\NOTE{
    در بخش بعد، خواهید دید که بهتر است برای حل مسائل شمارش با ترتیب، 
    \CROSSREF[از]{تابع مولد نمایی}
     تابع مولد نمایی
    بجای 
    \CROSSREF[تابع مولد]{توابع مولد در نظریه مجموعه‌ها}
     استفاده شود. اما توجه کنید در سوالاتی همچون سوال بالا که در آن‌ها تعداد اشیا (سکه‌ها) مجهول است،
    توابع مولد نمایی کمک زیادی نخواهند کرد.
}