\begin{PROBLEM}
    \p
    چند کلمه‌ی ۴ حرفی با حروف کلمه‌ی 
    $PAPAYA$
    می‌توان ساخت؟ 

    \SOLUTION[پاسخ به کمک تابع مولد و جایگشت خطی با اعضای تکراری]{
        \p
        ابتدا اهمیت ترتیب حروف در کلمات خروجی را نادیده می‌گیریم.
        برای سادگی در این پاسخ، کلماتی که ترتیب حروف در آن‌ها بی‌اهمیت است را
        «دسته حروف»
        می‌نامیم.
        بنابراین، با استفاده از روش 
        \CROSSREF[تابع مولد]{توابع مولد در نظریه مجموعه‌ها}
        ، می‌دانیم تعداد حالات انتخاب
        $4$
        حرف از حروف داده شده، برابر است با ضریب
        $x^4$
        در:
        $$\underbrace{(1 + x^1 + x^2 + x^3)}_{A}\times\underbrace{(1 + x^1 + x^2)}_{P}\times\underbrace{(1 + x^1)}_{Y}$$
        \p
        با توجه به نحوه شبیه‌سازی مسئله به تابع مولد، می‌دانیم که هر دسته حروف $4$ حرفی،
        متناظر حاصل ضرب جملاتی از عوامل (هر پرانتز و محتویاتش یک عامل محسوب می‌شوند) بالاست.
        می‌توان این حاصل ضرب را به شکل زیر نشان داد که در آن،
        $x^a$
        جمله‌ی استفاده شده از عامل
        $A$،
        $x^p$
        جمله‌ی استفاده شده از عامل
        $P$ و
        $x^y$
        جمله‌ی استفاده شده از عامل
        $Y$
        است:
        $$x^ax^px^y ; a+p+y=4$$
        \p
        گفته می‌شود که اگر ترتیب بی‌اهمیت باشد، آنگاه ضریب
        $x^4$،
        تعداد دسته حروف $4$ حرفی را نشان می‌دهد؛ دلیل این گفته آن است که ضریب
        $x^4$،
        درواقع تعداد عبارات
        $x^ax^px^y$
        به نحوی که
        $a+p+y=4$
        را نشان می‌دهد. دیدیم که این تعداد، به معنای تعداد دسته حروف $4$ حرفی است.
        \p
        حال نیاز داریم که ترتیب حروف در کلمات را درنظر بگیریم.
        برای این کار، کافیست بجای شمردن عبارات
        $$x^ax^px^y ; a+p+y=4$$
        در خروجی، مجموعِ تعدادِ جایگشت‌هایِ خطیِ متمایزِ دسته حروفِ متناظر با آن‌ها را محاسبه کنیم.
        واضح است که دسته حروف متناظر با
        $x^ax^px^y$
        دارای
        $a$
        حرف
        $A$،
        $p$
        حرف
        $P$ و
        $y$
        حرف
        $Y$
        است.
        بنابراین، 
        \CROSSREF[طبق]{جایگشت خطی با اعضای تکراری}
         تعداد جایگشت‌های خطی با اعضای تکراری،
        می‌دانیم تعداد جایگشت‌های خطی این حروف برابر است با:
        $$\frac{(a+p+y)!}{a!p!y!}$$
        \p
        پس طبق آنچه گفته شد، پاسخ مسئله برابر است با مجموع عبارت بالا،
        به ازای تمام $a,p,y$ها به شرطی که $a+p+y=4$. یعنی اگر پاسخ مسئله را $N$ بنامیم:
        $$N = \sum\limits_{\substack{a \in \{0,1,2,3\}\\p \in \{0,1,2\}\\y \in \{0,1\}\\a+p+y=4}} \frac{(a+p+y)!}{a!p!y!} = 4! \times \sum\limits_{\substack{a \in \{0,1,2,3\}\\p \in \{0,1,2\}\\y \in \{0,1\}\\a+p+y=4}} \frac{1}{a!p!y!} = 38$$
    }

    \SOLUTION[پاسخ به کمک تابع مولد نمایی]{
        \p
        برای هر یک از حروف، یک 
        \CROSSREF{تابع مولد نمایی}
         در نظر می‌گیریم. با توجه به تعداد هر کدام از حروف در این کلمه، تابع مولد آن به دست می‌آید.
        $P(x)$
            تابع مولد حرف $P$،
            $A(x)$
            تابع مولد حرف $A$،
            $Y(X)$
            تابع مولد حرف $Y$ و 
            $G(x)$
                تابع مولد نهایی است:
            \begin{align*}
                P(x) &= 1 + x + \frac{x^{2}}{2!}\\
                A(x) &= 1 + x + \frac{x^{2}}{2!} + \frac{x^{3}}{3!}\\
                Y(x) &= 1 + x\\
                \Rightarrow G(x) &= P(x)A(x)Y(x)    
            \end{align*}
        \p
        اگر ضریب 
        $\frac{x^{n}}{n!}$
            در بسط
            $G(x)$
            با 
            $a_{n}$
            برابر باشد، تعداد کل کلمات ۴ حرفی با 
            $a_{4}$
            برابر است. با محاسبه‌ی بسط، 
            $a_{4}$
            برابر ۳۸ به دست می‌آید.
    }
\end{PROBLEM}

\NOTE{
    دو پاسخ سوال فوق را مقایسه کنید.
    این مقایسه می‌تواند دید بسیار خوبی نسبت به علت صدق پاسخی که به کمک تابع مولد نمایی بدست می‌آید
    حاصل کند:
    \p
    در روش اول، دیدیم که پس از استفاده از تابع مولد،
    نیاز داشتیم که هر جمله‌ی
    $$x^n; n=\sum\limits_{i} c_i$$
    را تقسیم بر
    $\prod\limits_i c_i$
    کنیم
    که
    $c_i$ها
    تعداد اشیا هم‌جنس در محصول بودند.
    حال به تفاوت تابع مولد و تابع مولد نمایی دقت کنید
    (
        \CROSSREF[به یاد آورید]{تابع مولد نمایی}
         که در فصل نظریه مجموعه‌ها ذکر شد، تابع مولد دنباله
    $a_n$
    همان تابع مولد نمایی دنباله
    $\frac{a_n}{n!}$ است).
    همانطور که مشاهده می‌کنید، استفاده از تابع مولد نمایی بجای تابع مولد، در روشی یکسان برای شمارش،
    هم‌ارز اضافه کردن این تقسیم بر $\prod\limits_i c_i$
    در پاسخ روش معادله سیاله است.
    \p
    همچنین در روش اول، علاوه بر تقسیم بر
    $\prod\limits_i c_i$،
    نیاز داشتیم پاسخ را در
    $n!$
    نیز ضرب کنیم.
    دقت کنید که در روش شمارش به کمک توابع مولد نمایی، پاسخ مسئله ضریب عبارت
    $\frac{x^n}{n!}$
    است نه $x^n$.
    مشخص است که برای رسیدن به ضریب عبارت
    $\frac{x^n}{n!}$،
    کافیست ضریب عبارت
    $x^n$
    را در
    $n!$
    ضرب کنیم که این همان چیزی است که نیاز داشتیم.
    \p
    متوجه شدید که استفاده از 
    \CROSSREF{تابع مولد نمایی}
    ، چیزی نیست جز
    \CROSSREF[ضرب]{جایگشت خطی با اعضای تکراری}
     تعداد جایگشت‌های خطی با اعضای تکراری
     ، 
     \CROSSREF[در]{توابع مولد در نظریه مجموعه‌ها}
      پاسخ روش تابع مولد.
}