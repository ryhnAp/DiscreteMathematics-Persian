\begin{PROBLEM}
    \p
    به چند طریق می‌توان ۱۰
    سیب نامتمایز را
    بین محمد، علی، امیر و احمد
    تقسیم کرد اگر:

    \begin{enumerate}
        \item 
        هر کدام بتوانند هر تعداد سیب (صفر یا بیشتر) دریافت کنند.
        \SOLUTION{
            \p
            محمد، علی، امیر و احمد را به ترتیب با شماره‌های ۱ تا ۴ شماره‌گذاری می‌کنیم.
            تعداد سیب‌هایی که به شخص $i$ام می‌رسد را با $x_i$ نشان می‌دهیم.
             مسئله یک معادله سیاله در مجموعه اعداد حسابی به شکل زیر است:
            $$x_1+x_2+x_3+x_4 = 10$$
             پاسخ مسئله برابر است با 
             \CROSSREF[تعداد جواب‌های این معادله]{تعداد پاسخ‌های معادله سیاله خطی با ضرایب واحد در مجموعه اعداد حسابی}
              :
            $$\frac{(10+(4-1))!}{10!(4-1)!}=\frac{13!}{10!3!}=286$$
        }

        \item 
        هر کدام حداقل یک سیب دریافت کنند.
        \SOLUTION{
            \p
            در ادامه نام‌گذاری بخش قبل، داریم:
            $$x_1 \geq 1; x_2 \geq 1; x_3 \geq 1; x_4 \geq 1$$
            بنابراین تعداد جواب‌های معادله برابر است با:
            $$\frac{(10 - (1+1+1+1))+(4-1))!}{(10 - (1+1+1+1))!(4-1)!}$$
            $$=\frac{9!}{6!3!}=84$$
        }

        \item 
        محمد حداقل ۳ سیب و احمد حداقل ۱ سیب دریافت کند.
        \SOLUTION{
            \p
            در ادامه نام‌گذاری بخش اول، داریم:
            $$x_1 \geq 3; x_4 \geq 1$$
            بنابراین تعداد جواب‌های معادله برابر است با:
            $$\frac{(10 - (3+1))+(4-1))!}{(10 - (3+1))!(4-1)!}$$
            $$=\frac{9!}{6!3!}=84$$
        }

        \item 
        محمد حداکثر ۱ سیب و احمد حداقل ۱ سیب دریافت کند.
        \SOLUTION{
            \p
            در ادامه نام‌گذاری بخش اول، داریم:
            $$x_1 \leq 1; x_4 \geq 1$$
            برای حل این مسئله، روی مقدار
            $x_1$
            حالت‌بندی می‌کنیم:
            \begin{enumerate}
                \item 
                $x_1 = 0$

                در این صورت معادله سیاله زیر را داریم:
                $$x_2+x_3+x_4 = 10; x_4 \geq 1$$
                که تعداد پاسخ‌های آن برابر است با:
                $$=\frac{11!}{9!2!}=55$$

                \item 
                $x_1 = 1$

                در این صورت معادله سیاله زیر را داریم:
                $$1 + x_2+x_3+x_4 = 10; x_4 \geq 1$$
                $$x_2+x_3+x_4 = 9; x_4 \geq 1$$
                که تعداد پاسخ‌های آن برابر است با:
                $$=\frac{10!}{8!2!}=45$$
            \end{enumerate}
            طبق 
            \CROSSREF[اصل جمع]{اصل جمع در آنالیز ترکیبی}
            ، پاسخ مسئله برابر است با:
            $$55 + 45 = 100$$
        }
    \end{enumerate}
\end{PROBLEM}
