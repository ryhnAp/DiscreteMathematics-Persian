\begin{EXTRA}{معادلات سیاله}
    \TARGET{توضیحات مقدماتی سر فصل معادله سیاله} 
    \p
    \FOCUSEDON{معادله سیاله}
    در ریاضیات،
    معادله‌ای چند جمله‌ای با متغیرهای صحیح
    (مجهولات فقط می‌توانند مقادیر صحیح اتخاذ کنند)
    است. شکل کلی این معادلات را می‌توان به شکل زیر نمایش داد
    که در آن، تنها ‌$x_i$ها مجهول هستند
    (ضرایب و توان‌ها می‌توانند هر مقداری حقیقی داشته باشند): 
    $$\sum\limits_{i=1}^{n} \sum\limits_{j=1}^{d_i} ({a_{i,j}} \times {x_i}^{j}) = s$$
    \p
به عنوان مثال در معادله سیاله‌ی غیر خطی 
$$7x_1^3 + 8x_2^2 + 4x_2 + 5x_3^4 = 55$$
داریم:
\begin{align*}
    s &= 55\\
    a_{1, 3} &= 7\\
    a_{2, 2} &= 8\\
    a_{2, 1} &= 4\\
    a_{3, 4} &= 5
\end{align*}


    \p
        در معادلات سیاله، معمولا یکی از سوالات زیر مطرح است:
        \begin{enumerate}
            \item 
            آیا این معادله (در بازه‌ای محدود یا نامحدود) پاسخ دارد؟
            \item 
            این معادله (در بازه‌ای محدود یا نامحدود) چند پاسخ دارد؟
            \item 
            آیا قادر به محاسبه‌ی تمام پاسخ‌های معادله به صورت تئوری هستیم؟
            \item 
            آیا قادر به محاسبه‌ی حداقل یکی از پاسخ‌های معادله به صورت تئوری هستیم؟
            \item 
            آیا با استفاده از کامپیوتر، قادر به محاسبه‌ی تمام پاسخ‌های مسئله خواهیم بود؟
            \item 
            برای محاسبه‌ی تمام پاسخ‌های مسئله با استفاده از کامپیوتر، به چه میزان از منابع محاسباتی نیاز داریم؟
        \end{enumerate}
\end{EXTRA}