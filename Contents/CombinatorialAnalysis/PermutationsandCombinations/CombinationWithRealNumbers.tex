\begin{EXTRA}{
    ترکیب با مقادیر حقیقی
}
\p
حال که امکان محاسبه فاکتور مقادیر حقیقی را پیدا کردیم، می‌توانیم ترکیب مقادیر
حقیقی را نیز محاسبه کنیم. با وجود اینکه ما عملگر ترکیب
را با مقادیر حسابی تعریف کردیم و نحوه محاسبه آن را نیز با مقادیر حسابی بدست آوردیم،
اما رابطه‌ی یکسانی برای محاسبه ترکیبات مقادیر حقیقی استفاده می‌شود.
\p
مثال)
مقدار عبارت زیر را برحسب $\pi$ محاسبه کنید:
$$3 \choose -\frac{5}{2}$$

\SOLUTION{
    $${3 \choose -\frac{5}{2}} = \frac{3!}{-\frac{5}{2}!\times \frac{11}{2}!} = \frac{6}{\frac{4\pi}{15} \times \frac{10395\pi}{64}} = 0.1385\pi^{-2}$$
}
\p
همانطور که احتمالا تا الان به آن فکر کرده‌اید، درک مفهوم ترکیب مقادیر حقیقی دشوار و پیچیده است.
\p
یک نکته کاربردی آن است که برای محاسبه ترکیبات مقادیر حسابی از مقادیر حقیقی، نیازی به
محاسبه‌ی فاکتور مقادیر گویا نداریم.
شاید برایتان جالب باشد، با اینکه فاکتور مقادیر صحیح منفی تعریف نشده است، اما محاسبه‌ی
ترکیب مقادیر حسابی از مقادیر صحیح منفی ممکن است و این عبارات، دارای مقادیر صریح می‌باشند.
در این بخش، با چند مثال ساده، نحوه مواجهه با ترکیب مقادیر حسابی از مقادیر حقیقی را نشان خواهیم داد.

\p
مثال)
مقادیر صریح عبارات داده شده را محاسبه کنید:

    \begin{enumerate}
        \item 
        $5.2 \choose 2$

        \SOLUTION{
            $${5.2 \choose 2} = \frac{5.2!}{2!3.2!} = \frac{5.2\times 4.2\times 3.2!}{2!3.2!} = 10.92$$
        }

        \item 
        $-3 \choose 3$

        \SOLUTION{
            $${-3 \choose 3} = \frac{-3!}{3!\times-6!} = \frac{-3\times -4\times -5\times -6!}{3!\times-6!} = -10$$
        }

        \item 
        $-3.21 \choose 2$

        \SOLUTION{
            $${-3.21 \choose 2} = \frac{-3.21!}{2!\times-5.21!} = \frac{-3.21\times -4.21\times -5.21!}{2!\times-5.21!} = 13.5141$$
        }

        \item 
        $-\frac{7}{2} \choose -\frac{1}{2}$

        \SOLUTION{
            $${-\frac{1}{2} \choose -\frac{7}{2}} = {-\frac{1}{2} \choose 3} = \frac{-0.5!}{3!\times-3.5!} = -0.3125$$
        }
    \end{enumerate}

\p
مثال)
ترکیب از مقادیر صحیح منفی را بر حسب ترکیب از مقادیر حسابی و ضرایب صحیح نشان دهید.
    
    \SOLUTION{
        \begin{align*}
            {-n \choose r} &= \frac{(-n)(-n-1)(-n-2)...(-n-r+1)}{r!}\\
             &= \frac{(-1)^r (n)(n-1)...(n+r-1)}{r!}\\
             &= \frac{(-1)^r (n+r-1)!}{r!(n-1)!} = (-1)^r {n+r-1 \choose r}
        \end{align*}
    }

\end{EXTRA}
