\begin{EXTRA}{
    فاکتور مقادیر حقیقی و موهومی
}
\p
شاید برایتان جالب باشد اگر بدانید عملگر فاکتور ($!$)
نه فقط برای اعداد حسابی، بلکه برای اعداد حقیقی و حتی موهومی نیز تعریف شده است.
قابل توجه است که در مجموعه کل اعداد حقیقی و موهومی، تنها اعداد صحیح منفی هستند
که فاکتور تعریف نشده دارند. اثبات این موضوع دشوار نیست.
برای تعریف این عملگر، می‌دانید:
$$x! = x \times (x-1)!$$
پس:
$$(x-1)! = \frac{x!}{x}$$
درواقع فاکتور مقدار صفر، که گاها از آن به عنوان «استثنا» یا «حالت خاص» یاد می‌شود،
حاصل همین جابجایی ساده است:
$$0! = \frac{1!}{1} = 1$$
\p
از همین نوشتار برای مقادیر صحیح منفی استفاده می‌کنیم:
$$(-1)! = \frac{0!}{0} = \frac{1}{0}$$
که یک مقدار تعریف نشده است. بنابراین مقدار $(-1)!$ در مجموعه اعداد حقیقی تعریف نمی‌شود.
با منطقی مشابه می‌توان نشان داد فاکتور هیچ یک از اعداد صحیح منفی دیگر نیز تعریف نمی‌شود.
به عنوان مثال، مقدار $-3$ را درنظر بگیرید:
$$(-3)! = \frac{(-2)!}{-2} = \frac{(-1)!}{-2\times-1} = \frac{0!}{-2\times-1\times 0} = \frac{1}{0}$$
\p
درباره نحوه محاسبه مقدار فاکتور برای اعداد حقیقی یا موهومی، در مطلبی به نام 
«یک دیدگاه جدید بر فاکتور‌های اعداد حقیقی منفی و موهومی\LTRfootnote{Factorials of real negative and imaginary numbers -
A new perspective (\href{https://www.ncbi.nlm.nih.gov/pmc/articles/PMC4247832/}{https://www.ncbi.nlm.nih.gov/pmc/articles/PMC4247832/})}»
مباحث جالبی مطرح شده است. همچنین تاریخچه کوتاهی درمورد پیدایش این تفکر پیچیده را دربر می‌گیرد.
\p
در آن مطلب، برای اعداد حقیقی $x>-1$ آورده شده است:
$$x! = \prod(x) = \int_0^1 (-ln(t))^x dt = \int_0^\infty t^x e^{-t} dt = \Gamma(x+1)$$
با این رابطه، و دانش اینکه:
$$\prod(x) = x\times \prod(x-1)$$
قادر به محاسبه فاکتور تمام اعداد حقیقی و موهومی خواهیم بود.
جهت مطالعه بیشتر، به مقاله معرفی شده رجوع کنید.
\p
به عنوان یک نمونه پرکاربرد، خوب است اگر بدانید:
$$(-\frac{1}{2})! = \sqrt{\pi}$$
دقت کنید که برای محاسبه‌ی 
فاکتور اعداد گویا، تنها نیاز است مقدار فاکتور اعداد را در بازه‌ای به طول $[0,1)$ بدانیم، چرا که دیگر مقادیر از طریق آن‌ها، به سادگی قابل محاسبه‌اند.
مثال‌های زیر را درنظر بگیرید:
\p
مثال) مقدار عبارات داده شده را بر حسب
$\pi$
بیابید.
\begin{enumerate}
    \item 
    $\frac{1}{2}!$

    \SOLUTION{
        $$\frac{1}{2}! = \frac{1}{2} \times (\frac{1}{2}-1)! = \frac{1}{2} \times -\frac{1}{2}! = \frac{1}{2} \sqrt{\pi}$$
    }
    \item 
    $\frac{7}{2}!$
    
    \SOLUTION{
        $$\frac{7}{2}! = \frac{7}{2} \times \frac{5}{2} \times \frac{3}{2} \times \frac{1}{2}! = \frac{7 \times 5 \times 3}{2^3} \times \frac{1}{2} \sqrt{\pi}$$
    }
    \item 
    $-\frac{5}{2}!$
    
    \SOLUTION{
        $$-\frac{5}{2}! = \frac{-\frac{1}{2}!}{-\frac{3}{2} \times -\frac{1}{2}}  = \frac{4\sqrt{\pi}}{3}$$
    }
\end{enumerate}

\end{EXTRA}