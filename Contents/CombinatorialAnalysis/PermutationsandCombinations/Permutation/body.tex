\SUBSECTION{ترتیب و جایگشت}

\begin{DEFINITION}
    \TARGET{تعریف جایگشت}
    \p
    به هر روش قرار گرفتن چند شیء در کنار یک‌دیگر یک 
    \FOCUSEDON{جایگشت}
    از این اشیاء گفته می‌شود.
\end{DEFINITION}

\p
در مسائل ترکیبیاتی معمولا «تعداد جایگشت‌ها» مدنظر است
و گاهی از مواقع به اشتباه از واژه «جایگشت» بجای «تعداد جایگشت‌ها» استفاده می‌شود.

\begin{DEFINITION}
    \TARGET{تعریف جایگشت خطی}
    \p
    به هر روش قرار گرفتن چند شیء «به صورت خطی» (در یک صف) در کنار یک‌دیگر، یک
    \FOCUSEDON{جایگشت}
    \FOCUSEDON{خطی}
    از این اشیاء گفته می‌شود.
\end{DEFINITION}

\p
معمولا منظور از جایگشت، جایگشت خطی است مگر آن که نوع متفاوت جایگشت ذکر شود.

\begin{PROBLEM}[تعداد جایگشت‌های خطی]
    \p
    تعداد جایگشت‌های خطی یک مجموعه $n$ عضوی را بدست آورید.

    \SOLUTION{
        \p
        برای به دست آوردن یک جایگشت خطی از این مجموعه، نیاز داریم
        بین اعضای مجموعه و
        $n$
        جایگاه متمایز (شماره گذاری شده)،
        یک تناظر ارائه دهیم (اعضای مجموعه را از $1$ تا $n$ شماره گذاری کنیم).
        برای این کار از جایگاه اول شروع می‌کنیم و در گام
        $i$ام،
        برای جایگاه
        $i$ام
        یک عضو از اعضای انتخاب نشده در مجموعه را انتخاب می‌کنیم.
        مشخص است که در گام
        $i$ام،
        فارغ از انتخاب‌های قبلی،
        $n-i+1$
        عضو در مجموعه هستند که انتخاب نشده‌اند
        ($i-1$ عضو توسط $i-1$ اشغال شده‌اند).
        پس طبق 
        \CROSSREF[اصل ضرب]{اصل ضرب در آنالیز ترکیبی} 
        ، تعداد جایگشت‌های خطی ممکن برای این مجموعه برابر است با:
        $$n \times (n-1) \times (n-2) \times \dots \times 2 \times 1 = \prod\limits_{i=1}^n (n-i+1) = \prod\limits_{i=1}^n i = n!$$
    }
\end{PROBLEM}

\begin{THEOREM}
    \TARGET{اصل تعداد جایگشت‌های خطی با تعداد عضو مشخص}
    \p
    تعداد جایگشت‌های خطی ممکن برای یک مجموعه $n$
    عضوی، برابر 
    $n!$
    است.
\end{THEOREM}


\begin{DEFINITION}
    \p
    به هر روش قرار گرفتن
    $n$
    شیء دور یک دایره، یک 
    \FOCUSEDON{جایگشت}
    \FOCUSEDON{دوری}
    از این
    $n$
    شیء
    گفته می‌شود.
    اگر یک آرایش از دوران آرایش دیگری به دست آید، آن‌گاه این دو آرایش را هم‌ارز می‌دانیم.
\end{DEFINITION}

\NOTE{
    توجه شود که تغییر جهت چرخش به دور دایره موجب تمایز می‌شود. به عنوان مثال
    12345
    و
    34512
    نامتمایز اما
    12345
    و
    54321
    متمایز هستند.}

\begin{PROBLEM}[تعداد جایگشت‌های دوری]
    \p
    تعداد جایگشت‌های دوری یک مجموعه $n$ عضوی را بدست آورید.

    \SOLUTION{
        \p
        طبق 
        \CROSSREF[اصل تقارن]{اصل تقسیم (تقارن) در آنالیز ترکیبی} 
        ، می‌توان نتیجه گرفت
        تعداد جایگشت‌های دوری $n$ شیء متمایز برابر $(n-1)!$ است؛
        چرا که اگر یک جایگاه از حلقه‌ی جایگشت دوری را نقطه شروع درنظر بگیریم،
        به تعداد
        $n!$
        جایگشت خطی خواهیم داشت.
        با توجه به اینکه هر 
        $n$تا
        از آن‌ها حاصل دوران یک جایگشت هستند
        (هر بار یکی از اعضا در نقطه شروع قرار می‌گیرد و همان دنباله تکرار می‌شود)،
        در نتیجه
         تعداد حالات متمایز برابر است با :
        $$\frac{n!}{n} = (n-1)!$$
    }
\end{PROBLEM}

\begin{THEOREM}
    \TARGET{تعداد جایگشت‌های دوری برای مجموعه معلوم عضوی}
    \p
    تعداد جایگشت‌های دوری ممکن برای یک مجموعه $n$
    عضوی، برابر 
    $(n-1)!$
    است.
\end{THEOREM}


\begin{DEFINITION}
    \p
    هر انتخاب با ترتیب
    $r$
    عنصر از یک مجموعه
    $n$
    عضوی
    (یک جایگشت خطی بر زیرمجموعه‌ای 
    $r$
    عضوی از یک مجموعه
    $n$
    عضوی)،
    یک
    \FOCUSEDON{$r$-}
    \FOCUSEDON{ترتیب}
    از مجموعه
    $n$
    عضوی است.
    \FOCUSEDON{
    ترتیب 
    $r$
    از 
    $n$}
    به معنای تعداد
    $r$-ترتیب‌های ممکن
    از یک مجموعه $n$ عضوی
    بوده که آن را با 
    $P(n,r)$ نشان می‌دهیم. 
\end{DEFINITION}

\begin{THEOREM}
    \p
    اگر $n$ و $r$ اعدادی حسابی باشند به قسمی که 
    $r\leq n$، داریم:
    $$P(n,r) = n \times (n-1) \times (n-2) \times ... \times (n-r+1) = \frac{n!}{(n-r)!}$$
\end{THEOREM}

\NOTE{یک جایگشت خطی از یک مجموعه $n$ عضوی، درواقع یک $n$-ترتیب از آن مجموعه است.}

\subfile{./example1.tex}