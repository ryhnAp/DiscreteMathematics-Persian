\begin{PROBLEM}
    \p
    با حروف عبارت
    $mississippi$،
    ساختن چند کلمه متمایز با هر یک از شرایط زیر ممکن است
    (تمام حروف ملزم به استفاده شدن هستند)
    ؟
  
    \begin{enumerate}
      \item 
      بدون هیچ شرطی.
      
      \SOLUTION{
        \p
        توجه کنید که در حروفی که در اختیار داریم،
        ۴ حرف $s$،
        ۴ حرف $i$،
        ۲ حرف $p$
        و یک حرف $m$
        وجود دارد. بنابر 
        \CROSSREF[جایگشت تکراری]{تعداد جایگشت‌های خطی متمایز با اعضای تکراری}
        ، پاسخ مسئله برابر است با:
            $$\frac{11!}{2!4!4!} = 34650$$
      }
  
      \item 
      حروف
      $p$
      در مجاورت یکدیگر قرار نگیرند.
      
      \SOLUTION{
        \p
        بدون درنظر گرفتن حروف
        $p$،
        تمام کلمات ممکن را مانند حالت اول می‌سازیم. تعداد این حالات برابر است با:
            $$\frac{9!}{4!4!} = 630$$
        \p
        حال به ازای هر یک از کلمات ۹ حرفی ساخته شده،
        باید دو حرف $p$
        را اضافه کنیم به نحوی که در مجاورت یکدیگر قرار نگیرند.
            $$\_ m \_ i \_ s \_ s \_ i \_ s \_ s \_ s \_ i \_ i \_$$
        همانطور که در نمونه‌ی بالا می‌بینید، فارغ از جایگشت ۹ حرف اولیه،
        ۱۰
        جایگاه بین این ۹ حرف موجود است که حروف
        $p$
        می‌توانند در این جایگاه‌ها قرار بگیرند. توجه کنید که هر دو حرف
        $p$
        نباید به یک جایگاه تخصیص یابند چرا که در این صورت، در مجاورت هم قرار گرفته و
        شرط مسئله را نقض می‌کند. بنابراین، تعداد حالات چینش حروف 
        $p$
        برابر است با انتخاب دو جایگاه از این ۱۰ جایگاه:
            $$\binom{10}{2} = 45$$
        طبق 
        \CROSSREF[اصل ضرب]{اصل ضرب در آنالیز ترکیبی}
        ، پاسخ مسئله برابر است با:
            $$\frac{9!}{4!4!} \binom{10}{2} = 630 \times 45 = 28350$$
      }
  
      \item 
      حداقل دو حرف
      $s$
      در مجاورت یکدیگر باشند.
      
      \SOLUTION{
        \p
        برای بدست آوردن پاسخ این مسئله، از 
        \CROSSREF[اصل متمم]{اصل متمم در آنالیز ترکیبی}
         استفاده می‌کنیم.
        بنابراین تعداد عبارات ساخته شده که در آن هیچ دو حرف
        $s$
        در مجاورت یکدیگر قرار ندارند را بدست آورده و از تعداد کل عبارات ممکن
        کم می‌کنیم. تعداد کل عبارات ممکن را در قسمت اول همین سوال بدست آوردیم
        که این مقدار برابر 
        $34650$
        بود.
        \p
        برای بدست آوردن تعداد عبارات فاقد دو
        $s$
        مجاور، از روشی مشابه قسمت دوم استفاده می‌کنیم.
        بدون درنظر گرفتن حروف
        $s$،
        تمام کلمات ممکن را می‌سازیم. تعداد این حالات برابر است با:
            $$\frac{7!}{4!2!} = 105$$
        حال به ازای هر یک از کلمات ۷ حرفی ساخته شده،
        باید چهار حرف $s$
        را اضافه کنیم به نحوی که در مجاورت یکدیگر قرار نگیرند.
            $$\_ m \_ i \_ i \_ i \_ p \_ p \_ i \_$$
        مشابه قسمت دوم مسئله، تعداد حالات چینش حروف 
        $s$
        برابر است با انتخاب چهار جایگاه از ۸ جایگاه:
            $$\binom{8}{4} = 70$$
        طبق 
        \CROSSREF[اصل ضرب]{اصل ضرب در آنالیز ترکیبی}
        ، تعداد عبارات فاقد دو 
        $s$
        مجاور برابر است با:
            $$\frac{7!}{4!2!} \binom{8}{4} = 105 \times 70 = 7350$$
        \p
        طبق 
        \CROSSREF[اصل متمم]{اصل متمم در آنالیز ترکیبی}
        ، پاسخ مسئله برابر است با:
            $$34650 - 7350 = 27300$$
      }
  
      \item 
      حداقل دو حرف
      $s$
      در مجاورت یکدیگر یا دو حرف
      $i$
      در مجاورت یکدیگر باشند.
      
      \SOLUTION{
        \p
        از 
        \CROSSREF[اصل شمول و عدم شمول]{اصل شمول و عدم شمول}
         استفاده می‌کنیم. مجموعه‌ی
        $I$
        را مجموعه عباراتی که دارای حداقل دو حرف
        $i$
        در مجاورت یکدیگرند و مجموعه‌ی
        $S$
        را مجموعه عباراتی که دارای حداقل دو حرف
        $s$
        در مجاورت یکدیگر هستند درنظر بگیرید. در این صورت داریم:
            $$|S \cup I| = |S| + |I| - |S \cap I|$$
        \p
        اگر
        $A$
        را مجموعه‌ی تمام عبارات ممکن بدون شرط درنظر بگیریم،
        طبق اصل 
        \CROSSREF[متمم]{اصل متمم در آنالیز ترکیبی}
         داریم:
            $$|S \cup I|$$
            $$= (|A| - |(A - S)|) + (|A| - |(A - I)|) - (|A| - |(A - S \cap I)|)$$
            $$= |A| - |(A - S)| - |(A - I)| + |(A - (S \cap I))|$$
            $$= |A| - |S'| - |I'| + |SI'|$$
        که در آن،
        $S'$
        مجموعه عبارات فاقد دو حرف
        $s$
        در مجاورت یکدیگر،
        $I'$
        مجموعه عبارات فاقد دو حرف
        $i$
        در مجاورت یکدیگر و
        $SI'$
        مجموعه عبارات فاقد دو حرف
        $s$
        یا دو حرف
        $i$
        در مجاورت یکدیگر است.
        با توجه به تقارن موجود در مسئله برای دو حرف
        $s$
        و
        $i$
        (هر دو چهار بار تکرار شدند):    
            $$|S| = |I|, |S'| = |I'|$$
        \p
        با توجه به قسمت اول و سوم همین سوال، داریم:
            $$|S'| = |I'| = 7350, |A| = 34650$$
        \p
        برای بدست آوردن مقدار 
        $|SI'|$
        درست مانند قسمت سوم همین سوال عمل می‌کنیم. در ابتدا
        حروف غیر از
        $s$ و $i$
        را به کار می‌گیریم:
            $$\frac{3!}{2!} = 3$$
            $$\_ m \_ p \_ p \_$$
        سپس حروف
        $i$
        را در جایگاه‌ها قرار می‌دهیم
        (مانند قبل، هیچ دو حرف
        $i$
        در مجاورت هم و در یک جایگاه قرار نمی‌گیرند):
            $$\binom{4}{4} = 1$$
            $$\_ i \_ m \_ i \_ p \_ i \_ p \_ i \_$$
        سپس حروف
        $s$
        را در جایگاه‌ها قرار می‌دهیم
        (مانند قبل، هیچ دو حرف
        $s$
        در مجاورت هم و در یک جایگاه قرار نمی‌گیرند):
            $$\binom{8}{4} = 70$$
        \p
        طبق 
        \CROSSREF[اصل ضرب]{اصل ضرب در آنالیز ترکیبی}:
            $$|SI'| = \frac{3!}{2!} \binom{4}{4} \binom{8}{4} = 3 \times 1 \times 70 = 210$$
        پس داریم:
            $$|S \cup I| = |A| - |S'| - |I'| + |SI'|$$
            $$= 34650 - 7350 - 7350 - 210 = 19740$$
      }
  
    \end{enumerate}
\end{PROBLEM}