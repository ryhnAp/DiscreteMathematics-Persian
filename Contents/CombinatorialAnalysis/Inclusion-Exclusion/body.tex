\SECTION{اصل شمول و عدم شمول}

\begin{DEFINITION}
    \TARGET{اصل شمول و عدم شمول}
    \p
    \FOCUSEDON{اصل شمول و عدم شمول:}
    اگر بتوان فضای حالات عملی را به دو فضای
    $A_1$
    و
    $A_2$
    تقسیم کرد به نحوی که این دو فضا امکان اشتراک در اعضایشان را داشته باشند،
    آنگاه تعداد اعضای فضای حالت کل برابر است با :
    $$|A_1 \cup	A_2| = |A_1| + |A_2| - |A_1 \cap A_2|$$
\end{DEFINITION}

\NOTE{همانطور که در توضیحات مربوط به اصل جمع نیز گفته شد، آن اصل
فقط زمانی قابل استفاده است که حالات مختلف انجام یک عمل از دو مسیر،
اشتراکی نداشته باشند. این اصل برای رفع این محدودیت ارائه شده است.
منطق این اصل بسیار ساده است. اگر حالتی از انجام کار، در دو مسیر مشترک باشد،
اگر از اصل جمع استفاده کنیم، این حالت دو بار شمرده می‌شود. برای حل این ضعف،
به سادگی، این تعداد را یکبار از نتیجه کل کم می‌کنیم تا به تعداد حالات یکتا برسیم.}

\begin{THEOREM}
    \p
    \FOCUSEDON{تعمیم اصل شمول و عدم شمول}
     را می‌توان به شکل زیر نوشت:
    $$|\bigcup\limits_{i=1}^n A_i| = \sum\limits_{k=1}^n (-1)^{k+1} (\sum\limits_{1 \leq i_1 < \dots < i_k \leq n} |\bigcap\limits_{j \in \{i_1,...,i_k\}} A_j|)$$
\end{THEOREM}
\p
تساوی بالا، معادل است با:
$$|A_1 \cup A_2 \cup \dots \cup A_n| = |A_1| + |A_2| + \dots |A_n|$$
$$- |A_1 \cap A_2| - |A_1 \cap A_3| - \dots - |A_1 \cap A_n| - |A_2 \cap A_3| \dots |A_{n-1} \cap A_n|$$
$$+ |A_1 \cap A_2 \cap A_3| + |A_1 \cap A_2 \cap A_4| + \dots |A_{n-2} \cap A_{n-1} \cap A_{n}|$$
$$\dots$$
$$+ (-1)^{n+1} |A_1 \cap A_2 \cap \dots \cap A_n|$$

\subfile{./example1.tex}
\subfile{./example2.tex}
\subfile{./example3.tex}
\subfile{./example4.tex}
\subfile{./example5.tex}
\subfile{./example6.tex}

\subfile{./Derangement/body.tex}