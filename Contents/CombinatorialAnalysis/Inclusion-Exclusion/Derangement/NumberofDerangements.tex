\begin{PROBLEM}[تعداد پریش‌ها]
  \p
    تعداد پریش‌های یک مجموعه
    $n$
    عضوی را بدست آورید.

    \SOLUTION{
      \p
        عملگر
        $P(i)$
        را 
        \CROSSREF[تعداد جایگشت‌هایی]{تعریف جایگشت خطی}
         از مجموعه موردنظر تعریف می‌کنیم
        به نحوی که در این جایگشت‌ها، حداقل
        $i$
        عضو در جایگاه اصلی خود قرار داشته باشند.
        اگر پاسخ مسئله (تعداد پریش‌ها) را
        $N$
        و تعداد کل جایگشت‌های غیر پریش در این مجموعه را
        $A$
        بنامیم، طبق 
        \CROSSREF[اصل متمم]{اصل متمم در آنالیز ترکیبی}
         داریم:
          $$N = P(0) - A$$
        همچنین طبق 
        \CROSSREF[اصل شمول و عدم شمول]{اصل شمول و عدم شمول}
         داریم:
          $$A = P(1) - P(2) + P(3) - ... + (-1)^{n-1} P(n) = \sum\limits_{i=1}^n (-1)^{i-1} P(i)$$
          $$\Rightarrow N = \sum\limits_{i=0}^n (-1)^i P(i)$$
        \p
        برای بدست آوردن مقدار
        $P(i)$،
        ابتدا 
        $i$
        عضو از مجموعه را انتخاب کرده و آن‌ها را در جایگاه اصلی خود قرار می‌دهیم.
        حال می‌توانیم به مسئله به چشم 
        \CROSSREF[تعداد]{جایگشت خطی با اعضای تکراری}
         جایگشت‌های
        $n - i$
        عضوی باقیمانده نگاه کنیم
        (جایگاه‌های اشغال شده و اشیا قرار داده شده را نادیده می‌گیریم).
        طبق 
        \CROSSREF[اصل ضرب]{اصل ضرب در آنالیز ترکیبی}
        ، 
        $P(i)$
        برابر حاصل ضرب تعداد حالات انتخاب
        $i$
        عضو و تعداد 
        \CROSSREF[جایگشت‌های]{جایگشت خطی با اعضای تکراری}
         مجموعه جدید می‌باشد:
          $$P(i) = \binom{n}{i} (n-i)! = \frac{n!}{(n-i)! i!} (n-i)! = \frac{n!}{i!}$$
        \p
        بنابراین داریم:
          $$N = \sum\limits_{i=0}^n (-1)^i P(i) = \sum\limits_{i=0}^n (-1)^i \frac{n!}{i!} = n! \sum\limits_{i=0}^n \frac{(-1)^i}{i!} $$
    }
\end{PROBLEM}

\begin{THEOREM}
  \TARGET{تعداد پریش‌های یک مجموعه معلوم عضوی}
  \p
    تعداد پریش‌های یک مجموعه 
    $n$
    عضوی برابر است با:
    $$n! \sum\limits_{i=0}^n \frac{(-1)^i}{i!}$$
\end{THEOREM}