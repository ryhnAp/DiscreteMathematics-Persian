\begin{PROBLEM}[تعداد توابع پوشا]
  \p
  تعداد توابع پوشا از یک مجموعه
  $m$
  عنصری به یک مجموعه
  $n$
  عنصری
  را بیابید.
    
    \SOLUTION{
      \p
      عملگر
      $P(i)$
      را تعداد توابع از مجموعه
      $m$
      عضوی به مجموعه
      $n$
      عضوی تعریف می‌کنیم به نحوی که حداقل
      $i$
      عضو از مجموعه مقصد پوشانده نشوند.
      اگر پاسخ مسئله (تعداد توابع پوشا) را
      $N$
      و تعداد کل توابع غیر پوشا از مجموعه
      $m$
      عضوی به مجموعه
      $n$
      عضوی را
      $A$
      بنامیم، طبق 
      \CROSSREF[اصل متمم]{اصل متمم در آنالیز ترکیبی}
       داریم:
        $$N = P(0) - A$$
      همچنین طبق 
      \CROSSREF[اصل شمول و عدم شمول]{اصل شمول و عدم شمول}
       داریم:
        $$A = P(1) - P(2) + P(3) - ... + (-1)^{n-1} P(n) = \sum\limits_{i=1}^n (-1)^{i-1} P(i)$$
      
      \p
      برای بدست آوردن مقدار
      $P(i)$،
      ابتدا 
      $i$
      عضو از مجموعه مقصد به عنوان اعضای پوشانده نشده
      انتخاب می‌کنیم و آن‌ها را نادیده می‌گیریم.
      سپس تعداد کل توابع ممکن از مجموعه
      $m$
      عضوی مبدا به مجموعه
      $n - i$
      عضوی مقصد جدید را محاسبه می‌کنیم.
      طبق 
      \CROSSREF[اصل ضرب]{اصل ضرب در آنالیز ترکیبی}
      ،
      $P(i)$
      برابر حاصل ضرب تعداد حالات انتخاب
      $i$
      عضو و تعداد توابع بر روی مجموعه مقصد جدید می‌باشد:
        $$P(i) = \binom{n}{i} (n-i)^m$$
      \p
      بنابراین داریم:
      \begin{align*}
        N &= P(0) - A = P(0) - \sum\limits_{i=1}^n (-1)^{i-1} P(i) = \sum\limits_{i=0}^n (-1)^i P(i)\\
        &= \sum\limits_{i=0}^n (-1)^i \binom{n}{i} (n-i)^m
      \end{align*}
    }
\end{PROBLEM}

\begin{THEOREM}
  \TARGET{تعداد توابع پوشا از مبدا معلوم عضوی به مقصد معلوم عضوی}
  \p
  تعداد توابع پوشا از یک مجموعه 
  $m$
  عضوی به یک مجموعه
  $n$
  عضوی برابر است با:
  $$\sum\limits_{i=0}^n (-1)^i \binom{n}{i} (n-i)^m$$
\end{THEOREM}