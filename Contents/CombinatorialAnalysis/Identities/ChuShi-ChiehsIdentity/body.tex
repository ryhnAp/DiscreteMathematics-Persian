\SUBSECTION{اتحاد چوشی‌چی}

\begin{PROBLEM}[اتحاد چوشی‌چی]
  \p
  فرض کنید
  $n,r$
  اعداد صحیح نامنفی باشند و
  $r \leq n$.
  اتحاد زیر را ثابت کنید:
      $$\binom{n+1}{r+1} = \sum\limits_{j=r}^n \binom{j}{r}$$

  \SOLUTION[پاسخ با تعمیم اتحاد پاسکال]{
    \p
      طبق 
      \CROSSREF[اتحاد پاسکال]{اتحاد پاسکال برای اعداد حسابی}
       می‌دانیم:
      \begin{align*}
        &&& \binom{n-1}{r-1} + \binom{n-1}{r} &&=&& \binom{n}{r} \\
        &\Rightarrow&& \binom{n-1}{r-1} &&=&& \binom{n}{r} - \binom{n-1}{r}\\
        &\Rightarrow&& \binom{j}{r} &&=&& \binom{j+1}{r+1} - \binom{j}{r+1}
      \end{align*}
      پس:
      \begin{align*}
        \binom{r}{r} &= \binom{r+1}{r+1} &&-&& \binom{r}{r+1}\\
        \binom{r+1}{r} &= \binom{r+1+1}{r+1} &&-&& \binom{r+1}{r+1}\\
        \binom{r+2}{r} &= \binom{r+2+1}{r+1} &&-&& \binom{r+2}{r+1}\\
        &\vdots\\
        \binom{n-1}{r} &= \binom{n-1+1}{r+1} &&-&& \binom{n-1}{r+1}\\
        \binom{n}{r} &= \binom{n+1}{r+1} &&-&& \binom{n}{r+1}\\
        \Huge{+} \rule{0.15\textwidth}{0.8pt} &\rule{0.55\textwidth}{0.8pt} \span\span\span\span\span\span\span\span \\
        \sum\limits_{j=r}^n \binom{j}{r} &= \binom{n+1}{r+1} 
      \end{align*}

  }
    
  \SOLUTION[پاسخ از روش دوگانه شماری]{
      \p
      از 
      \CROSSREF[دوگانه شماری]{توضیح روش ترکیبیاتی دوگانه شماری}
       استفاده می‌کنیم. می‌خواهیم یک رشته باینری به طول
      $n+1$
      حاوی
      $r+1$
      بیت
      $1$
      بسازیم. تعداد حالات ساختن این رشته را به دو طریق می‌توان شمرد :
      \begin{enumerate}
      \item 
      تعداد
      $r+1$
      بیت را از بین
      $n+1$
      بیت موجود انتخاب کرده و آن‌ها را با
      $1$
      و مابقی بیت‌ها را با
      $0$
      مقداردهی می‌کنیم:
          $$\binom{n+1}{r+1}$$

      \item 
      ابتدا بیت مربوط به آخرین مقدار
      $1$
      مشاهده شده را انتخاب می‌کنیم. با توجه به اینکه ملزم به داشتن
      $r+1$
      بیت دارای مقدار
      $1$
      هستیم، جایگاه آخرین مقدار
      $1$
      می‌تواند
      $j+1$امین
      بیت باشد به نحوی که :
      $r+1 \leq j+1 \leq n+1$.
      حال بیت‌های اول تا
      $j$ام
      را درنظر بگیرید. در این رشته‌ی
      $j$ بیتی،
      ملزم به انتخاب
      $r$
      بیت برای
      $1$
      بودن هستیم که به
      $\binom{j}{r}$
      حالت صورت می‌گیرد. حال برای رسیدن به جواب مسئله، طبق 
      \CROSSREF[اصل جمع]{اصل جمع در آنالیز ترکیبی}
      ، باید
      تعداد حالات ساختن رشته را به ازای تمام مقادیر ممکن
      $j$
      با هم جمع کنیم. بنابراین پاسخ مسئله برابر است با :
          $$\sum\limits_{j=r}^n \binom{j}{r}$$
      \end{enumerate}
      \p
      با توجه به یکتا بودن پاسخ مسئله‌ی طرح شده، مقدار بدست آمده از دو روش برابر است،
      پس داریم :
      $$\binom{n+1}{r+1} = \sum\limits_{j=r}^n \binom{j}{r}$$
  }
\end{PROBLEM}

\begin{THEOREM}
  \TARGET{اتحاد چوشی‌چی برای اعداد حسابی}
  \p
  فرض کنیم
  $n,r$
  اعداد صحیح نامنفی باشند و
  $r \leq n$،
  آنگاه طبق
  \FOCUSEDON{اتحاد}
  \FOCUSEDON{چوشی‌چی}
  داریم:
    $$\binom{n+1}{r+1} = \sum\limits_{j=r}^n \binom{j}{r}$$
\end{THEOREM}