\SUBSECTION{اتحاد واندرموند}

\begin{PROBLEM}[اتحاد واندرموند]
  \p
  اثبات کنید اگر
  $m,n,r$
  اعداد صحیح نامنفی باشند و
  $r \leq n$
  یا
  $r \leq m$
  برقرار باشد، داریم:
  $$\binom{n+m}{r} = \sum\limits_{k=0}^r \binom{m}{r-k} \binom{n}{k}$$

  \SOLUTION{
    \p
    برای اثبات این اتحاد از 
    \CROSSREF[دوگانه شماری]{توضیح روش ترکیبیاتی دوگانه شماری}
     استفاده می‌کنیم. دو گروه از افراد داریم.
    گروه اول و دوم به ترتیب شامل
    $m$ و $n$
    نفر هستند.
    قصد داریم تعداد حالات انتخاب
    $r$
    نفر از این دو گروه را پیدا کنیم به نحوی که تعداد افراد انتخاب شده از هر گروه
    فاقد اهمیت باشد. برای این مقصود، دو شمارش زیر ممکن است :
    \begin{enumerate}
      \item 
      گروه‌بندی را فراموش کرده و از بین
      $m+n$
      نفر،
      $r$
      نفر را انتخاب می‌کنیم :
        $$\binom{m+n}{r}$$

      \item 
      ابتدا تصمیم می‌گیریم چه تعداد از
      $r$
      نفر انتخابی از گروه اول باشند. این تعداد که می‌تواند مقدار صفر تا
      $r$
      بگیرد را 
      $k$
      می‌نامیم. بنابراین باید 
      $k$
      عضو از بین
      $n$
      عضو گروه اول و
      $r-k$
      عضو باقیمانده را از بین
      $m$
      نفر گروه دوم انتخاب کنیم. طبق اصل 
      \CROSSREF[ضرب]{اصل ضرب در آنالیز ترکیبی}
       داریم :
        $$\binom{n}{k} \binom{m}{r-k}$$
      با توجه به 
      \CROSSREF[اصل جمع]{اصل جمع در آنالیز ترکیبی}
      ، برای رسیدن به تعداد حالات کل، باید مقدار بالا را برای
      تمام مقادیر
      $k$
      با هم جمع کنیم :
        $$\sum\limits_{k=0}^r \binom{n}{k} \binom{m}{r-k}$$
    \end{enumerate}
    \p
    با توجه به یکتا بودن پاسخ مسئله، مقدار به دست آمده از هر دو روش یکسان می‌باشند :
      $$\binom{n+m}{r} = \sum\limits_{k=0}^r \binom{m}{r-k} \binom{n}{k}$$
  }
\end{PROBLEM}

\begin{THEOREM}
  \TARGET{اتحاد واندرموند برای اعداد حسابی}
  \p
    \FOCUSEDON{اتحاد واندرموند:}
    اگر
    $m,n,r$
    اعداد صحیح نامنفی باشند و
    $r \leq n$
    یا
    $r \leq m$
    باشد، داریم:
    $$\binom{n+m}{r} = \sum\limits_{k=0}^r \binom{m}{r-k} \binom{n}{k}$$
\end{THEOREM}