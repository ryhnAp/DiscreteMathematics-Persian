\SUBSECTION{ضرایب چندجمله‌ای}

\begin{PROBLEM}[قضیه ضرایب دوجمله‌ای]
  \p
  نشان دهید
  اگر
  $n$
  یک عدد صحیح نامنفی باشد، داریم:

  $$(x+y)^{n} = \sum\limits_{j=0}^n \binom{n}{j} x^{n-j}y^j$$

  \SOLUTION{
    \p
    برای خوانایی بهتر، می‌نویسیم:
      $$(x+y)^{n} = \prod\limits_{i=1}^n (x+y)_i$$
    که در آن، اندیس‌ها هیچ معنایی نداشته و فقط برای آسان‌تر کردن اشاره به عامل‌های مختلف
    عبارت اضافه شده‌اند.
    \p
    حاصل عبارت بالا برابر با حاصل جمع تعدادی جمله می‌باشد که هر جمله از ضرب یک
    $x$ یا $y$
    از عامل
    $i$ام
    ($(x+y)_i$)
    به ازای 
    $1 \leq i \leq n$
    است. بنابراین جمله‌های به شکل
    $x^{n-j}y^j$
    زمانی مشاهده می‌شوند که از
    $j$
    تعداد عوامل، عبارت
    $y$
    و از 
    $n - j$
    تعداد باقی‌مانده، عبارت
    $x$
    انتخاب شود. 
    تعداد تکرار این جمله در حاصل عبارت،
    برابر است با
    $\binom{n}{j}$
    ؛
    چرا که برای ساختن یک جمله به شکل 
    $x^{n-j}y^j$،
    نیاز است از
    $j$ تا
    از عوامل، متغیر
    $y$
    و از مابقی
    متغیر
    $x$
    برداریم.
    بنابراین تعداد تکرار جمله 
    $x^{n-j}y^j$
    که ضریب این عبارت در خروجی است، برابر است با:
      $$\binom{n}{j}$$
    با لحاظ کردن نتیجه به دست آمده، برای تمام جمله‌های ممکن، داریم:

      $$(x+y)^{n} = \sum\limits_{j=0}^n \binom{n}{j} x^{n-j}y^j$$
  }
\end{PROBLEM}

\begin{THEOREM}
  \TARGET{قضیه ضرایب دوجمله‌ای برای اعداد صحیح}
  \p
  \FOCUSEDON{ضرایب دوجمله‌ای:}
    اگر
    $n$
    یک عدد صحیح نامنفی باشد، داریم:

    $$(x+y)^{n} = \sum\limits_{j=0}^n \binom{n}{j} x^{n-j}y^j$$
\end{THEOREM}

\NOTE{اگر جملات حاصل از بسط دوجمله‌ای را بر مبنای توان یکی از جملات مرتب کنیم، آنگاه ضرایب دوجمله‌ای یک عبارت درجه $k$،
متناظر با مقادیر سطر
$k$ام (شروع از صفر)
مثلث پاسکال خواهند بود.}

\p
به سادگی می‌توان قضیه‌ی ضرایب دوجمله‌ای را به ضرایب چندجمله‌ای نیز تعمیم داد.
اثبات این تعمیم به عنوان
\CROSSREF[تمرین]{اثبات قضیه‌ی ضرایب چندجمله‌ای}
بر عهده شما قرار گرفته است.

\begin{THEOREM}
  \TARGET{قضیه ضرایب چندجمله‌ای برای اعداد صحیح}
  \p
  \FOCUSEDON{ضرایب چندجمله‌ای:}
  اگر
  $n$
  یک عدد صحیح نامنفی باشد، داریم:
    $$(\sum\limits_{i=1}^k x_i)^n = \sum\limits_{\sum\limits_{i=1}^k j_i = n} (\binom{n}{j_1,j_2,...,j_k} \prod\limits_{i=1}^{k} x_i^{j_i})$$
\end{THEOREM}

\subfile{./example1.tex}
\subfile{./example2.tex}
\subfile{./example3/body.tex}
\subfile{./example4/body.tex}

\subfile{./GeneralizedPolynomialCoefficients.tex}