\SUBSECTION{اصل متمم}

\begin{DEFINITION}
    \TARGET{اصل متمم در آنالیز ترکیبی}
    \p
    \FOCUSEDON{اصل متمم:}
    فرض کنید قصد انجام عملی مانند
    $A$
    را داشته باشیم.
    مجموعه حالات انجام عمل
    $A$
    را
    $A_t$
    می‌نامیم. فرض کنید با تعیین شرایط 
    $c$،
    این مجموعه حالات را به زیر مجموعه‌ی
    $A_c$
    محدود می‌کنیم.
    آنگاه داریم:
    $$A_c = A_t - A_{\lnot c}$$
    که در آن
    $A_{\lnot c}$
    مجموعه حالات انجام عمل
    $A$
    بدون شرایط
    $c$
    (با داشتن شرایط نقیض $c$)
    است.
    به زبان ساده‌تر، حالات انجام عملی با شرایطی، برابر است با کل حالات انجام عمل، مگر آن‌هایی که شرایط مذکور را نقض می‌کنند.
\end{DEFINITION}


\subfile{./example1.tex}
\subfile{./example2.tex}
