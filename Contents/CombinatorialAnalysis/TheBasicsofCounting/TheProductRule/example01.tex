\begin{PROBLEM}
    \p
چهار شهر
$A, B, C, D$
را متصور شوید که از 
$A$
به
$B$
دو مسیر
$a$
و
$b$
، از
$B$
به
$C$
یک مسیر
$c$
و از 
$C$
به
$D$
سه مسیر
 $e, d$
و
$f$
وجود دارد. چند مسیر متفاوت برای سفر از
$A$
به
$D$
وجود دارد به شرطی که اگر از مسیر
$b$
رد شویم، حق عبور از مسیر 
$f$
را نداشته باشیم؟
  \centerimage{0.5}{1.png}

        \SOLUTION{
            \p
           در این مسئله قادر به استفاده از اصل ضرب نیستیم زیرا عبور از مسیر 
        $b$   
           با توجه به شرطی که در مسئله ذکر شده، باعث کم شدن تعداد حالات پیشروی می‌شود. پاسخ این سوال با استفاده از ترکیب اصل ضرب و اصل جمع که در 
           \CROSSREF[ادامه‌ی کتاب]{اصل جمع در آنالیز ترکیبی} 
           با آن آشنا خواهید شد، به‌دست می‌آید.
           به زبان ساده، نیاز است مسئله را به دو حالت عبور از $b$
           و عدم عبور از $b$
           تقسیم کرده و پاسخ‌های دو حالت را با هم جمع کنیم:
           $$1\times 1 \times 3 + 1\times 1 \times 2 = 5$$   
        }
\end{PROBLEM}