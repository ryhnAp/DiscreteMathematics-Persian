\SUBSECTION{اصل ضرب}
\begin{DEFINITION}
    \TARGET{اصل ضرب در آنالیز ترکیبی}
    \p
    \FOCUSEDON{اصل ضرب:}
    اگر بتوان فرایندی را به دو قسمت متوالی تقسیم کرد و
    $n$
    حالت برای انجام قسمت اول و به ازای هر یک از این حالات،
    $m$
    حالت برای انجام قسمت دوم وجود داشته باشد،
    آنگاه
    $n \times m$
    حالت برای انجام شدن فرایند وجود دارد.
\end{DEFINITION}
    
\begin{THEOREM}
    \p
    \FOCUSEDON{تعمیم اصل ضرب:}
    اگر بتوان فرایندی را به
    $k$
    قسمت متوالی تقسیم کرد و
    به ازای هر دنباله‌ای از حالات رخداد $i-1$ گام اول،
    $n_i$
    حالت برای انجام قسمت
    $i$ام
    وجود داشته باشد،
    آنگاه
    $\prod\limits_{i=1}^n n_i$
    حالت برای انجام این فرایند وجود دارد.
\end{THEOREM}

\subfile{./example1/body.tex}

\p
به فرضیات اصل ضرب تعمیم‌یافته توجه کنید:
\NOTE{
    نه فقط مرحله‌ی
    $i-1$ام
    بلکه کل دنباله‌ی اعمال پیشین
    تأثیرگذار است، به نحوی که اگر عملی در ابتدای دنباله،
    باعث کم یا زیاد شدن تعداد حالات پیشروی در انتهای دنباله شود،
    قادر به استفاده از این اصل نیستیم.
}
\p
برای درک بهتر این نکته، به مثال زیر توجه کنید:
\subfile{./example01.tex}

\NOTE{
    نیازی نیست حالات انجام کار در مرحله‌ی
    $i$ام
    به ازای تمام دنباله‌های اعمال 
    $1$
    تا
    $i-1$ام
    یکسان باشد، بلکه تنها کافیست تعداد این حالات برابر باشد.
}
\p
برای درک بهتر این نکته، به مثال زیر توجه کنید:
\subfile{./example02.tex}

\subfile{./example2.tex}
\subfile{./example3/body.tex}
\subfile{./example4/body.tex}