\CHAPTER[./SetTheory.jpg]{نظریه مجموعه‌ها}{
    نظریه مجموعه‌ها شاخه‌ای از علم ریاضی است که به مطالعه مجموعه‌ها می‌پردازد.
    مجموعه‌ها به صورت کلی، به گردایه‌ای از اشیاء گفته می‌شود.
    این اشیاء می‌توانند هرچیزی باشند.
    مباحث مربوط به نظریه مجموعه‌ها بسیار گسترده می‌باشد.
    این فصل از کتاب هنوز کامل نشده و تنها قسمت‌هایی
    از آن که برای مباحث آینده نیاز است، نوشته شده است.
}{
    \href{https://pixabay.com/users/skitterphoto-324082}{Skitterphoto}
}

% \subfile{./Intro/body.tex}

% \subfile{./DeMorgansLawsandAlgebraofSets/body.tex}

% \subfile{./Category/body.tex}

% \subfile{./BijectionsandFunctions/body.tex}

% \subfile{./CountableandUncountableSets/body.tex}

% \subfile{./Infinity/body.tex}

% \subfile{./Paradoxes/body.tex}

\subfile{./Sequences/body.tex}

\subfile{./GeneratingFunctions/body.tex}

\subfile{./PigeonholePrinciple/body.tex}

\subfile{./RepresentationofRecursiveSequences/body.tex}

\subfile{./DoNotMakeMistakes/body.tex}

\subfile{./Problems/body.tex}

\subfile{./Exercises/body.tex}


