\begin{PROBLEM}
	\p
    در یک دنباله حسابی 
    $f_4 + f_{11} = 105$
    و 
    $f_3 + f_7 = 80$
    است.
    $f_1$
    و قدر نسبت را محاسبه کنید.    
     
    \SOLUTION{
    \p
می‌دانیم:
		$$a_n = a_1 + (n - 1)d$$
پس:
        \begin{align*}
            f_4 + f_{11} = 105 &\Rightarrow 2f_1 + (3 + 10)d = 105\\
            f_3 + f_7 = 80 &\Rightarrow 2f_1 + (2 + 6)d = 80\\
            &\Downarrow\\
            d &= 5\\
            f_1 &= 20   
        \end{align*}
    }
\end{PROBLEM}