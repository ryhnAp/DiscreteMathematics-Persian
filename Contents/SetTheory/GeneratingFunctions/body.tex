\SECTION{توابع مولد}

\begin{DEFINITION}
    \TARGET{توابع مولد در نظریه مجموعه‌ها}
    \p
    \FOCUSEDON{تابع مولد}
    تابعیست که برای نمایش دنباله‌ی 
    $a_{n}$ 
    به صورت ضرایبی از توان‌های 
    $x$
    از آن استفاده می‌شود، به طوری که ضریب 
    $x^{n}$
    برابر است با جمله‌ی
    $n$
    ام دنباله. تابع مولد برای دنباله‌ی 
    $a_{0}, a_{1}, a_{2}, ..., a_{n}, ... $
    برابر است با:
        $$G(x)= a_{0} + a_{1}x^{1} + a_{2}x^{2} + ... + a_{n}x^{n} + ... = \sum\limits_{n=0}^{+\infty} a_{n}x^{n}$$
\end{DEFINITION}
\p
برای مثال،
تابع مولد برای دنباله‌ی 
$1, 1, 1, 1, 1, 1$ 
به صورت زیر است:
    $$G(x)= 1 + x + x^2 + x^3 + x^3 + x^4 + x^5 = \frac{x^6-1}{x-1}$$
    $$x \neq 1$$

همچنین
تابع مولد برای دنباله‌ی 
$3, 3, 3, ...$
به صورت زیر است: 
    $$G(x)= \sum\limits_{n=0}^{+\infty} 3x^{n} = \frac{3}{1-x}$$
    $$|x| < 1$$

و برای دنباله‌ی 
$1, c, c^2, c^3, ...$
تابع مولد به صورت زیر است: 
    $$G(x)= \sum\limits_{n=0}^{+\infty} (cx)^{n} = \frac{1}{1-cx}$$
    $$|cx| < 1 , c \neq 0$$

\subfile{./example2.tex}
\subfile{./example3.tex}
\subfile{./UsefulGeneratingFunctions.tex}





\begin{DEFINITION}
    \TARGET{تابع مولد نمایی}
    \p
    \FOCUSEDON{تابع مولد نمایی}
    نسخه‌ای تغییر یافته از تابع مولد است که
    برای دنباله‌ی 
    $a_{n}$
    به صورت زیر تعریف می‌شود:
      $$G(x)= a_{0} + a_{1}(\frac{x}{1!}) + a_{2}(\frac{x^{2}}{2!}) + ... = \sum\limits_{n=0}^{+\infty} a_{n}(\frac{x^{n}}{n!})$$
\end{DEFINITION}
\p
درواقع تفاوت این تابع، با تابع مولد در آن است که در تابع مولد،
$a_k$
ضریب عبارت
$x^k$
بود ولی در تابع مولد نمایی، ضریب عبارت
$\frac{x^k}{k!}$
است.
این تابع برای
استفاده‌های خاصی که در آینده با آن‌ها روبرو خواهید شد،
تعریف شده است.

\NOTE{اگر بتوان ارتباط بین دو دنباله‌ی
$a_n$ و $b_n$
را به شکل $a_n = \frac{b_n}{n!}$
نوشت، آنگاه تابع مولد نمایی دنباله
$a_n$
همان تابع مولد دنباله
$b_n$
خواهد بود.}
\p
برای نمونه،
تابع مولد نمایی برای دنباله‌ی 
$0!, 1!, 2!, ..., n!, ...$ 
به صورت زیر است:
    $$G(x)= \sum\limits_{n=0}^{+\infty} (n!)(\frac{x^{n}}{n!}) = 1 + x + x^{2} + ... = \frac{1}{1-x}$$	

همچنین تابع مولد برای دنباله‌ی 
$1, k, k^{2}, ..., k^{n}, ...$
نیز به صورت زیر است: 
    $$G(x)= \sum\limits_{n=0}^{+\infty} \frac{(kx)^{n}}{n!} = 1 + \frac{kx}{1!} + \frac{(kx)^{2}}{2!} + ... = e^{kx}$$